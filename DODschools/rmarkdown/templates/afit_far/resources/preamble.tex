%-----------------ADDED BY SLV---------------------------
%added by SLV for acronymns
\usepackage[printonlyused, withpage]{acronym}  % facilitates the use of acronyms

%%%%%%%%%%%%%%%%%%%%%%%%%%%%%%%%%%%%%%
\makeatletter       %%suppresses hyperref warning: Bookmark level unknown for first usage of [appendix]
\providecommand*{\toclevel@appendix}{0}
\makeatother
%%%%%%%%%%%%%%%%%%%%%%%%%%%%%%
\usepackage{natbib} %This is for APA style
%\usepackage[bookmarks=true,hidelinks]{hyperref}
%%%%%%%%%%%%%% Multiple bibliographies%%%%%%%%%
\usepackage{multibib}
\newcites{DIS}{Binary Measures References}
\newcites{MAL}{Malware References}
%% !!!! compile, then open CMD, cd to folder where .tex files are, and execute 'bibtex DIS' everytime you update.

%%%%%%%%%%%%%%%%%%%%%%%%%
\usepackage{url}
\usepackage{caption}
%added by SLV to suppress warnings under specific curcumstances
\usepackage{silence}
\WarningFilter[tbd]{latex}{tbd}
%added by SLV to suppress underfull hbox warning in bibliography
\usepackage{etoolbox}
\apptocmd{\sloppy}{\hbadness 10000\relax}{}{}

\setlength{\emergencystretch}{3pt} %% suppresses Overfull \hbox for minor overfull situations
\hyphenation{kas-per-sky} %%provides guidance on how to split words unknown to Latex, prevents Overfull hbox
\hyphenation{ran-som-ware}
\hyphenation{stan-o-vich}
\hyphenation{aut-on-o-mous}
\hyphenation{a-t-tem-p-ts}
\hyphenation{a-b-str-a-cts}
\hyphenation{MBR-Ransom-ware}
\hyphenation{Neshta-Virus}
\hyphenation{Spy-Banker}
\hyphenation{VikingAT}
\hyphenation{WORM-Brontok}
\hyphenation{WORM-Ramnit}
\hyphenation{WORM-Viking}
\hyphenation{ac-quir-ed}



\let\theoremstyle\relax
\usepackage{amsthm,lipsum}
\usepackage{amsmath}
%%%%%%%%%%%%%%%%%%%%%%%%%%%%
\makeatletter
\def\thm@space@setup{%
  \thm@preskip=.5cm
  \thm@postskip=\thm@preskip % or whatever, if you don't want them to be equal
}
\makeatother


%%%%%%%%%%%%%%%%%%%%%%%%


\usepackage[section]{placeins} %redefins section with “\FloatBarrier” inserted at the beginning
\usepackage[normalem]{ulem} %supports \sout for strike-out
\newcommand{\mycomment}[1]{}
\mycomment{ this is a
multiline
comment.}

\usepackage{booktabs}
\usepackage{longtable}
\usepackage{rotating}
\usepackage{subfig}
\usepackage{adjustbox}
\usepackage{array,multirow,graphicx}
\newcommand{\specialcell}[2][c]{%
  \begin{tabular}[#1]{@{}c@{}}#2\end{tabular}}
%%http://tex.stackexchange.com/questions/9796/how-to-add-todo-notes
\usepackage{xargs}

%%  http://tex.stackexchange.com/questions/32683/rotated-column-titles-in-tabular
\newcolumntype{Q}[2]{%
    >{\adjustbox{angle=#1,lap=\width-(#2)}\bgroup}%
    l%
    <{\egroup}%
}
\newcommand*\rot{\multicolumn{1}{Q{45}{1em}}}% no optional argument here, please!
%%%%%%%%%%%%%%%%%%%%%%%%%%%%%%%%%%%%%%%%%%%%%%%%%%%%%%
\newcommand{\smInfty}{\mbox{\tiny $\infty$}}
\newcommand{\smNeg}{\mbox{\tiny $-$}}
%%%%%%%%%%%%%%%%%%%%%%%%%%%%%%%%%%%%%%%%%%%%%%%%%%%%%%

\usepackage[pdftex,dvipsnames,table]{xcolor}  % Coloured text etc.
\definecolor{light-gray}{gray}{0.80}
\definecolor{light-green}{RGB}{204, 255, 255}
%%%%%%%%%%%%%%%%%%%%% tabular updates %%%%%%%%%%%%%%%%%%%%%%%%%%%%
\usepackage{tabu}

\usepackage{siunitx}  %allows table columns with numberic rounding/decimals S[round-precision=2]
%http://tex.stackexchange.com/questions/12703/how-to-create-fixed-width-table-columns-with-text-raggedright-centered-raggedlef
\newcolumntype{L}[1]{>{\raggedright\let\newline\\\arraybackslash\hspace{0pt}}m{#1}}
\newcolumntype{C}[1]{>{\centering\let\newline\\\arraybackslash\hspace{0pt}}c{#1}}
\newcolumntype{B}[1]{>{\centering\let\newline\\\arraybackslash\hspace{0pt}}b{#1}}
\newcolumntype{R}[1]{>{\raggedleft\let\newline\\\arraybackslash\hspace{0pt}}m{#1}}
\newcolumntype{P}[1]{>{\centering\arraybackslash}p{#1}} % Horizontal centering
\newcolumntype{M}[1]{>{\centering\arraybackslash}m{#1}} % Horizontal and Vertical centering
\newcommand{\rr}{\raggedright}
\newcommand{\tn}{\tabularnewline}
%%%%%%%%%%%%%%%%%%%%%%%%%%%%%%
\makeatletter % changes the catcode of @ to 11
\renewenvironment{table}
    {\renewcommand{\arraystretch}{0.75}% vertical spacing of tables
     \@float{table}}
    {\end@float}
\makeatother % changes the catcode of @ back to 12
%%%%%%%%%%%%%%%%%%%%%%%%%%%%%%
\setcounter{tocdepth}{3}
\setcounter{secnumdepth}{3}
%%%%%%%%%% NOMENCLATURE %%%%%%%%%%%%%%%%%%%%%%%%%%%%%
%To generate new Nomenclature list, open a command window, cd to folder containing SLVaughanDissertationDraft.nlo and execute:
%Compile Latex
%"C:\Program Files\MiKTeX 2.9\miktex\bin\x64\makeindex.exe" SLVaughanDissertationDraft.nlo -s nomencl.ist -o SLVaughanDissertationDraft.nls
%Compile Latex again
\usepackage[intoc]{nomencl}
\renewcommand{\nomname}{List of Symbols, Abbreviations and Acronyms}
\makenomenclature
%%%%%%%%%%%% END OF NOMENCLATURE %%%%%%%%%%%%%%%%%%%%%%
\usepackage[]{quoting}
\quotingsetup{vskip=0pt}
\def\:{\hskip0pt}  %example: q\:-\:connected    >>> q-connected properly hyphenated.
\usepackage{tablefootnote}
\usepackage{pdflscape}
\usepackage{units}
\newcommand{\specialcellTwor}[2][c]{%
  \begin{tabular}[#1]{@{}c@{}}#2\end{tabular}}
%% Example: \specialcellTwor[t]{Data Set\\(\emph{\gls{\Gls{tv}}})}

%% color coding for Draft
\newcommand{\DraftColorBICA}{\color[rgb]{0.00,0.07,1.00}}
\newcommand{\DraftColorProspectus}{\color[rgb]{0.00,0.53,0.68}}
\newcommand{\DraftColorWorkingPaper}{\color[rgb]{0.00,0.46,0.29}}
\newcommand{\ReturnColorToBlack}{\color[rgb]{0.00,0.00,0.00}}

%\DeclareMathSizes{text size}{math text size}{superscript size}{subscript size}.
\newlength{\mytextsize}
\makeatletter
      \setlength{\mytextsize}{\f@size pt}
\makeatother
%\DeclareMathSizes{\mytextsize}{\mytextsize}{8}{8}

\newif\ifDraft
%\Drafttrue % comment out to hide color
%http://ctan.mackichan.com/macros/latex/contrib/hyperref/README.pdf
\usepackage[bookmarks=true,hidelinks,breaklinks]{hyperref}
\usepackage{breakcites}
\usepackage[capitalise]{cleveref}
\crefname{section}{section}{sections}
\crefname{equation}{equation}{equations}
\Crefname{figure}{Figure}{Figures}

\usepackage{enumitem}% http://ctan.org/pkg/enumitem
\newlist{question}{enumerate}{2}
\setlist{parsep=0pt,listparindent=\parindent}
\setlist[question,1]{label=\textbf{RQ-\arabic*}}
\setlist[question,2]{label=\textbf{RQ-\arabic{questioni}.\arabic*}}
\crefname{questioni}{}{}
\Crefname{questioni}{}{}
\crefname{questionii}{}{}
\Crefname{questionii}{}{}
%\crefname{questioni}{question}{questions}
%\Crefname{questioni}{Question}{Questions}
\SetLabelAlign{Center}{\hfil#1\hfil}
\SetLabelAlign{CenterWithParen}{\hfil(\makebox[1.0em]{#1})\hfil}
%%%%%%%%%%%%%%%%%%%%%%%%

%added by SLV for to reference description items
%http://texblog.org/2012/03/21/cross-referencing-list-items/
\makeatletter
\def\namedlabel#1#2{\begingroup
    #2%
    \def\@currentlabel{#2}%
    \phantomsection\label{#1}\endgroup
}
\makeatother
%Example:
%\item[\namedlabel{itm:rule1}{Rule 1}] Everything is easy






%added by SLV for Glossary %%%%%%%%%%%%%%%%%%%%%%%%%%%%
%glossaries must be leaded after hyperref
%\usepackage[toc,nowarn]{glossaries} %nowarn - disables warning that no \printglossary is found
% http://tex.stackexchange.com/questions/253950/refer-to-a-glossary-entry
\usepackage[toc,hyperfirst=true,entrycounter,numberedsection=autolabel]{glossaries}
\renewcommand*{\glsentrycounterlabel}{}
\renewcommand*{\glsautoprefix}{gls:}
\robustify{\gls}
%\renewcommand{\glstextformat}[1]{\textit{#1}} %This command makes all glossary references italicised

\newglossary[glignoredl]{ignored}{glignored}{glignoredin}{Ignored Glossary}

% Note that if you use the hyperref package, you will need to use \nohyperpage in
% the suffix to ensure that the hyperlinks work correctly. For example:
\glsSetSuffixF{\nohyperpage{f.}}
\glsSetSuffixFF{\nohyperpage{ff.}}
% must be used before \makeglossaries and have no effect if \noist is used.

\makeglossaries
%\glsdisablehyper %disables warnings that glossary has not been generated
%%%%%%%%%%%%%%%%%% end of glossary %%%%%%%%%%%%%%%%%%%%%%%%%%%%%
%Theorems added by SLV
%\newtheorem{theorem}{Theorem} has two parameters, the first one is the name of the environment that is defined, the second one is the word that will be printed, in boldface font, at the beginning of the environment.
%%%%%%%%%%%%%%%%%%%%%%%%%%%%%%%%
\theoremstyle{definition}
\newtheorem{example}{Example}[chapter]
\crefname{example}{Example}{Examples}

\theoremstyle{definition}
\newtheorem{mydefinition}{Definition}[chapter]
\crefname{mydefinition}{Definition}{Definitions}

\theoremstyle{definition}
\newtheorem{Bexample}{Example}[section]
\crefname{Bexample}{Example}{Examples}

\theoremstyle{definition}
\newtheorem{Bdefinition}{Definition}[section]
\crefname{Bdefinition}{Definition}{Definitions}

